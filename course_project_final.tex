\documentclass[conference]{IEEEtran}
\IEEEoverridecommandlockouts
% The preceding line is only needed to identify funding in the first footnote. If that is unneeded, please comment it out.
\usepackage{cite}
\usepackage{amsmath,amssymb,amsfonts}
\usepackage{algorithmic}
\usepackage{graphicx}
\usepackage{textcomp}
\usepackage{xcolor}
\def\BibTeX{{\rm B\kern-.05em{\sc i\kern-.025em b}\kern-.08em
    T\kern-.1667em\lower.7ex\hbox{E}\kern-.125emX}}
\begin{document}

\title{Course Project Final Report\\

}

\author{\IEEEauthorblockN{Clayton Tucker}
\IEEEauthorblockA{\textit{COSC325} \\
\textit{Ctrl Alt Delete}\\
Knoxville, US \\
ctucke24@vols.utk.edu}
\and
\IEEEauthorblockN{Todd Van Meter}
\IEEEauthorblockA{\textit{COSC 325} \\
\textit{Ctrl Alt Delete}\\
Knoxville, US \\
tvanmete@vols.utk.edu}
\and
\IEEEauthorblockN{Danyil Chuprynov}
\IEEEauthorblockA{\textit{COSC 325} \\
\textit{Ctrl Alt Delete}\\
Knoxvile, US \\
dchupryn@vols.utk.edu}
\and
\IEEEauthorblockN{Ian Henson}
\IEEEauthorblockA{\textit{COSC 325} \\
\textit{Ctrl Alt Delete}\\
Knoxvile, US \\
tkd995@vols.utk.edu}
}

\maketitle

\begin{abstract}
This document is a model and instructions for \LaTeX.
This and the IEEEtran.cls file define the components of your paper [title, text, heads, etc.]. *CRITICAL: Do Not Use Symbols, Special Characters, Footnotes, 
or Math in Paper Title or Abstract.
\end{abstract}

\begin{IEEEkeywords}
component, formatting, style, styling, insert
\end{IEEEkeywords}

\section{Introduction}
Predicting stock prices is a challenging yet critical task in finance due to market volatility and unpredictability. Machine learning provides powerful tools capable of identifying complex patterns within historical financial data, which can lead to more accurate stock price forecasts.

We aim to specifically examine the application of machine learning techniques to predict the stock prices of Berkshire Hathaway, a leading investment firm known for its diverse portfolio and market influence (Umer Haddi, 2023). Accurate prediction models for Berkshire Hathaway's stock could significantly assist investors in making informed decisions, managing investment risks, and capitalizing on market trends.

Utilizing the Berksire-Hathaway historical stock data from Kaggle (Umer Haddi, 2023), we aim to:
\begin{itemize}
    \item Analyze and pre-process historical data to identify predictive features.
    \item Implement and evaluate multiple ML models to determine their forecasting accuracy.
    \item Assess the practical effectiveness of ML in predicting Berkshire Hathaway’s stock performance.
\end{itemize}

Through these objectives, the project seeks to contribute to improving investment strategies and demonstrating the value of ML applications in financial market analysis.

\section{Data Exploration}

\subsection{The Dataset}

The dataset used in this study is the Berkshire Hathaway Stock Price Data, sourced from Kaggle (Haddi, 2023). This dataset includes historical stock information consisting of 1,671 daily samples collected from January 2, 2020, to July 29, 2024. Each sample contains eight features: open price, high price, low price, close price, adjusted close price, trading volume, log-transformed closing price, and date converted to an ordinal number format. The dataset is continuous and inherently balanced as it involves time-series data rather than categorical classes. Due to its nature, it requires special pre-processing techniques, such as stationary adjustments and log transformations, to ensure effective application of predictive models.

\subsection{Preprocessing}
Detailed preprocessing steps applied to the data included:
\begin{itemize}
    \item Filtering to include only data from January 2020 onwards to maintain consistency and relevance.
    \item Handling missing values by forward-filling daily gaps to maintain continuity, ensuring no temporal discontinuities in the dataset.
    \item Transforming the 'Close' price feature using a logarithmic scale to stabilize variance and manage heterogeneity of variance.
    \item Calculating the first-order differences on all features to achieve stationarity, a requirement for ARIMA modeling, confirmed through the Augmented Dickey-Fuller (ADF) test.
    \item Then a moving average and standard deviation features was calculated to improve accuracy, the latter was required for ARIMA.
\end{itemize}

The preprocessing steps above were critical to improving and ensuring the effectiveness of the predictive models employed in this study.

\subsection{Data Analysis}\label{AA}
\begin{itemize}
    \item \textbf{Overall Stock Price Trend}: Berkshire Hathaway's stock price exhibited an increasing trend over the analyzed period from January 2020 to July 2024. The closing price started around \$228 in early 2020 and rose significantly, reaching approximately \$438 in July of 2024, suggesting overall growth and positive long-term performance.
    \item \textbf{Seasonality and Stationarity}: Initial tests for stationarity ADF test indicated non-stationarity in raw closing prices (Test Statistic: -0.52, P-Value: 0.89). Thus, data transformations (log differences) were applied to make the dataset stationary, which is essential for effective ARIMA modeling.
    \item \textbf{Volume Insights}: The daily traded volume shows significant variability, indicating periods of increased investor activity. The high volatility in trading volume suggests events or news that significantly influenced investor behavior and stock prices.
    \item \textbf{Model Performance Comparison}: ARIMA and Random Forest models produced similar predictive accuracy (RMSE approximately 22.2 and 22.1, respectively), outperforming Linear Regression, which had notably higher errors (RMSE approximately 31.6). This confirms non-linearity in stock price data and highlights the effectiveness of non-linear models like ARIMA and Random Forest.
    \item \textbf{Error Analysis}: Both ARIMA and Random Forest displayed relatively low and similar absolute errors (~18), indicating that while there are predictive limitations, both models adequately captured short-term movements in stock prices.
    \item \textbf{Possible Overfitting with Random Forest}: Random Forest exhibited consistent predictions across test days (constant values), suggesting possible overfitting to the training data. While it resulted in low errors, the model’s predictions lacked variability, potentially limiting its usefulness in dynamic real-world trading scenarios.
\end{itemize}

\section{Baseline Model}

\section{Model Improvements}

\section{Distribution of Work}

\begin{thebibliography}{00}
\bibitem{b1} G. Eason, B. Noble, and I. N. Sneddon, ``On certain integrals of Lipschitz-Hankel type involving products of Bessel functions,'' Phil. Trans. Roy. Soc. London, vol. A247, pp. 529--551, April 1955.

\end{thebibliography}

\end{document}


